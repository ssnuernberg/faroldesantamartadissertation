\chapter{Introduction}
\label{ch:intro}

 The \textit{Praia Limpa Santa Marta} project represents an innovative response to a pressing environmental issue in the Farol de Santa Marta region, a renowned beach destination in Southern Brazil. Known for its stunning natural landscapes and thriving tourism industry, Farol de Santa Marta faces significant challenges due to the environmental impact of increasing tourist activity. This proposal outlines the development of a non-profit website dedicated to promoting the 'leave no trace' principle and facilitating sustainable practices within the community.

% Choose your own headings.

\section{Motivation}

The motivation for developing the \textit{Praia Limpa Santa Marta} project arises from a significant and urgent need within the community of Farol de Santa Marta. As a region celebrated for its stunning beaches and robust tourist activities, it faces environmental challenges due to the increased influx of tourists. These visitors, while beneficial economically, often leave behind waste, threatening the natural beauty and health of the local ecosystem.
    
    The cleanliness and beauty of Farol de Santa Marta's beaches are not only vital for the local ecosystem but also for the continued attraction of tourists. Tourism is a major economic driver for the community, providing livelihoods and supporting local businesses. Maintaining pristine beaches is crucial for sustaining this economic benefit, as littered and polluted areas could deter visitors and harm the local economy.
    
    A cleaner, well-maintained environment attracts more tourists, which in turn boosts the local economy. Increased tourism leads to more business for local shops, restaurants, and accommodation providers. Additionally, involving the community in conservation efforts can strengthen social bonds and promote a sense of unity and collective purpose. This collaborative spirit can have far-reaching positive effects, enhancing the overall quality of life in Farol de Santa Marta.
    
    Local stakeholders, including small business owners, environmental activists, and residents, have expressed a keen interest in addressing this issue. However, they lack the technical expertise and resources to establish an online platform that could effectively mobilize the community and manage conservation efforts. This gap presents a unique opportunity to leverage web development skills for social and environmental impact.


\section{Aim}

    The project aims to harness this local enthusiasm by providing a tailored web solution that not only raises awareness about the 'leave no trace' principle but also actively engages both locals and tourists in sustainability practices. By creating a platform where community members can learn about environmental preservation, sign up for beach clean-ups, and contribute financially through donations, \textit{Praia Limpa Santa Marta} will empower the community to take ownership of their environment and foster a culture of sustainability.

    This initiative is not just about cleaning up the beaches; it's about building a community-driven movement that ensures the long-term preservation of Farol de Santa Marta's natural landscapes, benefiting both the environment and the local economy reliant on tourism. Through this project, we aim to transform environmental concern into actionable change, making it a compelling venture for anyone passionate about conserving our natural world.

\section{Template Used}


    This project follows the guidelines of the CM3035 Advanced Web Development template titled "Project Idea Title 1: A non-profit web application". The project is structured as a non-profit web application, reflecting a strategic approach to combining web development skills with environmental conservation efforts. The chosen template supports the project's goals by providing a framework that facilitates the development of a user-centric, responsive, and scalable platform that can be easily maintained and adapted for future use.
