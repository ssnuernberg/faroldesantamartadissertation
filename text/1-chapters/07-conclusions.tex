\chapter{Conclusions and further work}
\label{ch:conclusions}

\section{Summary of Work}

The \textit{Farol de Santa Marta} project was undertaken with the primary objective of developing a user-friendly, secure, and scalable platform to support environmental conservation efforts in the Farol de Santa Marta community. The website aimed to engage local residents, tourists, environmental activists, and students in activities such as event participation, educational course enrollment, and donations to local conservation initiatives.

\section{Contributions}

The \textit{Farol de Santa Marta} project represents a significant contribution to the local community’s environmental conservation efforts. By providing an accessible and engaging platform, the project helps bridge the gap between community members and conservation activities. The originality of the project lies in its tailored approach to the specific needs of the Farol de Santa Marta community, combining best practices from global conservation platforms with a localized focus.

Moreover, the integration of educational resources and interactive features, such as course enrollment and event participation, not only raises awareness about environmental issues but also empowers users to take action. The platform’s design, which prioritizes user experience and accessibility, serves as a model for similar initiatives in other regions.

\section{Future Work}

While the \textit{Farol de Santa Marta} website successfully achieved its initial goals, there are several areas for potential improvement and extension:

\begin{itemize}
    \item \textbf{Expanded Educational Content}: Future iterations of the project could focus on expanding the range of educational materials available on the platform, including more courses on various environmental topics and involving more local experts in content creation.
    \item \textbf{Nonprofit Integration}: Strengthening partnerships with local nonprofits is crucial for fully realizing the donation system's potential. Future work should prioritize these collaborations to ensure that the platform remains compliant with local regulations and effectively channels donations to support conservation projects.
    \item \textbf{Better Implementation of Donation System}: The integration of a donation system is still in development. Future work should focus on fully implementing this feature, ensuring it is secure, user-friendly, and compliant with local regulations. This will involve collaborating with local nonprofits to streamline the process of receiving donations.
\end{itemize}


In conclusion, the \textit{Farol de Santa Marta} project has laid a strong foundation for ongoing and future environmental conservation efforts in the region. By addressing the areas for improvement and exploring potential extensions, the project can continue to evolve, making an even greater impact on the community and the environment.

%%% Local Variables:
%%% mode: latex
%%% TeX-master: "../../main"
%%% End:
