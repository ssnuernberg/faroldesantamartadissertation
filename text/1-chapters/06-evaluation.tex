\chapter{Evaluation}
\label{ch:evaluation}

% Choose your own headings.
\section{User Testing}

\subsubsection{Usability Testing}

Usability testing was conducted to assess how easily and effectively users could navigate and interact with the website. A group of test users, representing the target audience of the website—including local community members, tourists, environmental activists, and students—were asked to perform specific tasks such as registering on the platform, enrolling in a course, participating in an event, and making a donation.

\textbf{Methodology:}

\begin{itemize}
    \item \textbf{Task Scenarios:} Participants were given predefined scenarios to complete, such as signing up for a beach clean-up event or purchasing an eco-friendly product from the store.
    \item \textbf{Observation:} User interactions were observed, and notes were taken on any difficulties encountered, areas of confusion, and overall user satisfaction.
    \item \textbf{Surveys and Feedback:} After completing the tasks, users were asked to fill out a survey rating their experience on various aspects such as ease of use, navigation, and visual design. Additional feedback was gathered through open-ended questions to identify areas for improvement.
\end{itemize}

\textbf{Findings:}

The usability testing revealed that the majority of users found the website intuitive and easy to navigate. Some minor issues were identified, such as the need for clearer labeling on certain buttons and the simplification of the donation process. These insights led to refinements in the user interface, ensuring that the website better aligned with user expectations and needs.

\subsection{Agile Methodology and Iterative Feedback}

The development of the \textit{Farol de Santa Marta} website was guided by Agile methodology, which emphasizes iterative development, continuous user feedback, and flexibility in responding to change. This approach allowed the developer to engage users throughout the process, ensuring that the website met their needs and expectations.

\textbf{User Testing and Feedback Loop:}

During one of the user testing sessions, participants highlighted the need for a more visually engaging homepage. Specifically, users suggested adding an image that represents the essence of Farol de Santa Marta, to better capture the attention of visitors and convey the website's mission.

\textbf{Outcome:}

The inclusion of the image improved the aesthetics of the homepage, ultimately enhancing their overall experience. This iterative process of testing and refinement, core to Agile practices, proved essential in developing a user-centered platform that effectively supports the goals of the \textit{Farol de Santa Marta} project.

\section{Critical Analysis}

In this section, the results of the \textit{Farol de Santa Marta} project are critically analyzed in relation to the project's goals. The discussion focuses on what worked well throughout the development and implementation phases, as well as identifying areas that could be improved in future iterations.

\subsection{Analysis of Results in Relation to Project Goals}

The primary goals of the \textit{Farol de Santa Marta} project were to create a user-friendly, secure, and scalable platform that supports environmental conservation efforts in the local community. These objectives were largely met, as evidenced by the successful deployment of the website and positive feedback from user testing.

\textbf{Goal 1: User Engagement and Accessibility}

One of the key objectives was to ensure that the website was accessible to a wide audience, including local residents, tourists, environmental activists, and students. The usability testing indicated that the website is intuitive and easy to navigate, with users able to complete key tasks such as event registration, course enrollment, and donations with minimal difficulty. The implementation of a responsive design also ensured that the site could be accessed on a variety of devices, further supporting the goal of broad accessibility.

\textbf{Goal 2: Support for Environmental Initiatives}

The project aimed to facilitate environmental conservation efforts through features such as event management, educational resources, and a donation system. The system’s ability to handle event creation and management and course enrollment indicates that these goals were effectively met. User feedback confirmed that the platform successfully supports the community's conservation efforts by providing essential tools and resources.

\subsection{What Worked Well}

Several aspects of the project were particularly successful:

\begin{itemize}
    \item \textbf{Agile Development Process:} The use of Agile methodology allowed for continuous feedback and iterative improvements. This approach ensured that the project remained aligned with user needs and could adapt to changing requirements.
    \item \textbf{Responsive Design:} The decision to implement a responsive design from the outset ensured that the website provided a consistent and accessible experience across different devices. This contributed significantly to user satisfaction.
\end{itemize}

\subsection{Areas for Improvement}

Despite the successes, there are several areas where the project could be improved:

\begin{itemize}
    \item \textbf{Integration of Local Nonprofits:} The integration with local nonprofits for the donation process is still in development. Future efforts should focus on establishing partnerships with these organizations to fully realize the donation system's potential and ensure compliance with local regulations.
    \item \textbf{Expanded Educational Content:} While the courses page provides valuable resources, the content could be expanded to cover a wider range of environmental topics, potentially involving more local experts and educators in content creation.
    \item \textbf{Better implementation of donation:} The integration of a donation system is still in development.    
\end{itemize}

%%% Local Variables:
%%% mode: latex
%%% TeX-master: "../../main"
%%% End:
