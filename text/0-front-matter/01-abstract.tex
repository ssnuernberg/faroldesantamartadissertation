\chapter*{Abstract}

The \textit{Praia Limpa Santa Marta} project is a non-profit initiative aimed at addressing the growing environmental challenges faced by the coastal region of Farol de Santa Marta, a popular tourist destination in Southern Brazil. With the influx of visitors, the area has experienced significant environmental degradation, particularly in the form of waste and pollution left on its pristine beaches. This project seeks to develop a comprehensive web platform designed to promote the 'leave no trace' principle, engage the local community, and facilitate sustainable environmental practices. 

The proposed website will serve as a central hub for environmental education, community organization, and fundraising efforts. It will provide resources to educate both locals and tourists about the importance of preserving the natural beauty of the region. Additionally, the platform will enable users to participate in beach clean-up events, contribute to conservation projects through donations, and track the impact of their efforts. 

By leveraging digital tools, \textit{Praia Limpa Santa Marta} aims to transform individual environmental concern into collective action, fostering a culture of sustainability that benefits both the environment and the local economy. The project addresses the critical gap in technical expertise and organizational capacity among local stakeholders, providing them with the necessary resources to effectively manage and sustain conservation efforts. Ultimately, this initiative seeks to ensure the long-term preservation of Farol de Santa Marta's natural landscapes, making it a model for other communities facing similar challenges. 
 
Total Words: 9497

